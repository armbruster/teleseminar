
\documentclass[12pt,journal,compsoc]{IEEEtran}
%
% If IEEEtran.cls has not been installed into the LaTeX system files,
% manually specify the path to it like:
% \documentclass[12pt,journal,compsoc]{../sty/IEEEtran}


% *** CITATION PACKAGES ***
%
\ifCLASSOPTIONcompsoc
\else
\fi

\ifCLASSINFOpdf
\else
\fi


\usepackage{url}
\usepackage{multirow}

\hyphenation{Web-GL-Viewer Web-GL}

\usepackage{color}
\definecolor{lightgray}{rgb}{0.9,0.9,0.9}

\usepackage{listings} % Listings Code taken from: http://stackoverflow.com/questions/741985/latex-source-code-listing-like-in-professional-books
  \usepackage{courier}
 \lstset{
         basicstyle=\footnotesize\ttfamily, % Standardschrift
         %numbers=left,               % Ort der Zeilennummern
         numberstyle=\tiny,          % Stil der Zeilennummern
         %stepnumber=2,               % Abstand zwischen den Zeilennummern
         numbersep=5pt,              % Abstand der Nummern zum Text
         tabsize=2,                  % Groesse von Tabs
         extendedchars=true,         %
         breaklines=true,            % Zeilen werden Umgebrochen
         keywordstyle=\color{red},
    		frame=b,         
 %        keywordstyle=[1]\textbf,    % Stil der Keywords
 %        keywordstyle=[2]\textbf,    %
 %        keywordstyle=[3]\textbf,    %
 %        keywordstyle=[4]\textbf,   \sqrt{\sqrt{}} %
         stringstyle=\color{white}\ttfamily, % Farbe der String
         showspaces=false,           % Leerzeichen anzeigen ?
         showtabs=false,             % Tabs anzeigen ?
         xleftmargin=17pt,
         framexleftmargin=17pt,
         framexrightmargin=5pt,
         framexbottommargin=4pt,
         backgroundcolor=\color{lightgray},
         showstringspaces=false      % Leerzeichen in Strings anzeigen ?        
 }
 
   %\captionsetup[lstlisting]{singlelinecheck=false, labelfont={blue}, textfont={blue}}
  \usepackage{caption}
\DeclareCaptionFont{white}{\color{white}}
\DeclareCaptionFormat{listing}{\colorbox[cmyk]{0.43, 0.35, 0.35,0.01}{\parbox{0.485\textwidth}{\hspace{15pt}#1#2#3}}}
\captionsetup[lstlisting]{format=listing,labelfont=white,textfont=white, singlelinecheck=false, margin=0pt, font={bf,footnotesize}}



\begin{document}


\title{WebGL on Mobile Devices}


\author{William~Almnes,~\IEEEmembership{University~of~Oslo}\\
        Marko~Andjic,~\IEEEmembership{University~of~Oslo}\\
        Matthias~Armbruster~\IEEEmembership{University~of~Mannheim}\\  
        and~Paul~Steinhilber,~\IEEEmembership{University~of~Mannheim}% <-this % stops a space

\IEEEcompsocitemizethanks{\IEEEcompsocthanksitem William W. F. Almnes: wwalmnes@student.matnat.uio.no}
\IEEEcompsocitemizethanks{\IEEEcompsocthanksitem Marko Andjic: marko.andjic@usit.uio.no}
\IEEEcompsocitemizethanks{\IEEEcompsocthanksitem Matthias Armbruster: marmbrus@rumms.uni-mannheim.de}
\IEEEcompsocitemizethanks{\IEEEcompsocthanksitem Paul R. E. Steinhilber: ps@paulsteinhilber.de}
\IEEEcompsocitemizethanks{\IEEEcompsocthanksitem Daniel Sch\"{o}n: schoen@informatik.uni-mannheim (advisor)}
\thanks{Submitted just before April 23, 2012.}}


\markboth{Joint Teleseminar 2012}%
{Shell \MakeLowercase{\textit{et al.}}: Bare Demo of IEEEtran.cls for Computer Society Journals}

\IEEEcompsoctitleabstractindextext{%
\begin{abstract}
The abstract goes here.
\end{abstract}


\begin{IEEEkeywords}
WebGL, mobile computing, teleseminar.
\end{IEEEkeywords}}


\maketitle

\IEEEdisplaynotcompsoctitleabstractindextext
\IEEEpeerreviewmaketitle


\section{Introduction}
The technologies which allow people to interact with the world around them evolve constantly. Whereas it was common to simply display information in a static form in the early days of the web, at present an ever increasing amount of content is presented in dynamic and interactive ways.

In the effort to expand internet browser functionality to natively support three-dimensional graphics, HTML5, WebGL have been developed. Many web sites already offer functionalities which were previously only found in native applications, e.g. word processing using Google Docs \cite{googledocs} or creating presentation using 280 Slides \cite{280slides}, thus narrowing "the gap between them" \cite{Golubovic2011}. A native usage of 3d-functionality enhances these "web applications" even further.

% TODO Clear definition of AR 
"Augmented Reality (AR) has been defined broadly as combining real and computer-generated digital information into the user’s view of the physical and interactive real world in such a way that they appear as one environment, thus providing a bridge between digital information and the physical world" (\cite{Hoellerer2004,Klopfer2007,Vallino1998,Wellner1993} via \cite{Olsson2011b}). Application areas of AR can be as broad as ranging from health care \cite{Lui2011} to education \cite{Mannuss2011} to tourism \cite{Mulloni2011}. The topic has gained momentum thanks to the rise in smartphone usage. "To make the world itself the user interface [...] may revolutionize the way information is accessed and presented to people in the future \cite{Hoellerer2004,Wellner1993}.


% TODO change secondary source (newsweek) to primary source (apple keynote), i.e., add entry to .bib and change entry in .tex)
The rise of smartphones is growing with a fast and still accelerating pace, enabled ways of displaying information in a new way in a truly mobile context to many people \cite{Gartner2010,Nielsen2011}. They offer a much higher power than feature phones and even claim to offer the "real web" experience with "real browsers" \cite{informationweek2007}. There are differences, however, between the way information can be accessed from a desktop system and a mobile device, influenced by factors like screen size, processing power, and input methods.

% TODO change wording (method) --- maybe "technology for" ??
3d-support on mobile devices is still in an early phase. Adobe abandoned Flash in late 2011 \cite{Flash2011}, leaving WebGL as the main (method) of providing interactive 3d content on the mobile web, even if it is still in an initial phase.

This paper analyzes the status quo and potentials of WebGL on mobile devices (regarding performance and human-computer-interaction), and answers the following questions:

\begin{enumerate}
	\item \textbf{Are Augmented Reality applications possible on mobile devices using only the browser?}
	\item \textbf{Is 3d-augmentation of these applications possible using WebGL?}
	\item \textbf{How fast is WebGL on mobile devices?}
\end{enumerate}

The paper is structured as follows: Chapter 2 gives background information on relevant topics, i.e. augmented reality, WebGL, and human-computer-interation (HCI) with mobile devices and their evaluation criteria; Chapter 3 gives implementation details of the the WebGL environment used to capture the status quo; Chapter 4 presents the evaluation as well as limitations to this study; and Chapter 5 gives a summary of the results and presents future research opportunities.

\section{Background}

\subsection{WebGL}
In 2009, the \textit{WebGL Working Group} was formed with the mission to "bring hardware-accelerated graphics to the internet" \cite{Khronos2009}.  WebGL is a software library and extends Javascript to allow it to natively generate websites with 3D content. The first version of WebGL was released in 2011 \cite{Khronos2011}.

\subsubsection{Design}
"WebGL is based on OpenGL ES 2.0 and provides an API for 3D graphics. It uses the HTML5 canvas element and is accessed using Document Object Model interfaces. Automatic memory management is provided as part of the JavaScript language.... Renders of GPU, which muss support shader rendering" (Wikipedia)
It is based on Canvas 3D, which was developed at Mozilla \cite{Mozilla2007}.
"WebGL is a context of the canvas HTML element that provides a 3D computer graphics API without the use of plug-ins." (Wikipedia)

% Insert Table: WebGL support in major web browsers (see \cite{Golubovic2011}).
\subsubsection{Desktop WebGL}

% TODO: Make table of mobile web browser support for WebGL, see \cite{Golubovic2011})
\subsubsection{Mobile WebGL}
Although support for WebGL is available on the desktop computers within all major browsers WebGL (FireFox 4+, Safari 5.1+, Chrome 10+, Opera 12+), except of Microsoft's Internet Explorer \cite{Golubovic2011}, browsers on mobile devices supporting WebGL are still very rare.

\label{WebGLoniOS} Apple has added WebGL capabilities to iOS with iOS 4.2 \cite{deVries2011, iAdDocu}. However officially WebGL is only available to be used on Apple’s iAd platform \cite{deVries2011, cmarrin2011}. With an hack discovered by Nathan de Vries \cite{deVries2011} WebGL can also be enabled in a UIWebView. Using this way it is possible to build a custom WebGL viewer for iOS. However since this method of enabling WebGL on iOS requires the use of non-public APIs it can’t be made available to end users due to Apple's App Store Review Guidelines \cite{appstorereviewguidelines}. The built-in Safari browser on iOS does not support WebGL and due to Apple’s App Store Review Guidelines \cite{appstorereviewguidelines} third party browsers are required to use the WebKit rendering engine provided by Apple. So until Apple officially supports WebGL in Safari on iOS there is no way available for the end user to run WebGL content on iOS. 

As the desktop version of InternetExplorer does not support WebGL \cite{Golubovic2011} and Microsoft is considering WebGL "harmful" \cite{microsoftWebGLHarmful} we probably wont see WebGL support on Windows Phone 7 in the near future. 

With Firefox 4 a WebGL compatible browser is available on Android. % TODO: needs citation

Also RIM's tablet, the BlackBerry PlayBook, supports WebGL in web applications. % TODO: cite: http://devblog.blackberry.com/2012/02/playbook-native-webgl-development/ 


\subsubsection{Security Issues}
Although the support of WebGL seems to be widespread, there are also critical voices, with Microsoft probably being their most prominent spokesperson. Microsoft believes "that WebGL will likely become an ongoing source of hard-to-fix vulnerabilities" \cite{microsoftWebGLHarmful}, because WebGL allows direct access to the computers hardware from the web. WebGL security therefore relies on graphic card drivers and other third party components. \cite{microsoftWebGLHarmful}

Context, an information security consultancy, which recommends to disable WebGL in the browser, published and demonstrated two possible attack scenarios both initiated from a malicious website \cite{contextWebGL1, contextWebGL2}. They were not only able to perform a successful Denial of Service attack, which leads to operating system crashes and freezes of the desktop, but also to gather possible confidential information by stealing the content of the graphic memory with which they were able to reconstruct screenshots of the desktop. As stated by Context, these issues are inherent to the WebGL specification and can't be resolved without major changes in WebGL's architecture. Although there are countermeasures in development, which could resolve these issues, at the moment WebGL allows malicious programs access to the graphics hardware and software. \cite{contextWebGL1, contextWebGL2}

% These issues are inherent to the WebGL specification and would require significant architectural changes in order to remediate in the platform design. Fundamentally, WebGL now allows full (Turing Complete) programs from the internet to reach the graphics driver and graphics hardware which operate in what is supposed to be the most protected part of the computer (Kernel Mode). [http://www.contextis.com/resources/blog/webgl/]


% http://news.cnet.com/8301-30685_3-20071726-264/microsoft-declares-webgl-harmful-to-security/ == microsoftWebGLHarmful
% http://www.contextis.com/resources/blog/webgl/ == contextWebGL1
% http://www.contextis.com/resources/blog/webgl2/ == contextWebGL2
% http://blogs.technet.com/b/srd/archive/2011/06/16/webgl-considered-harmful.aspx


\subsubsection{Status Quo}

% TODO: Is there a comprehensive set of evaluation criteria in the literature?
\subsubsection{Evaluation criteria}
In the literature, WebGL implementations on mobile devices have been technically evaluated using various methods and range from checking support of official WebGL desktop browser examples to typed array conformance tests to performance tests \cite{Golubovic2011}. 

% TODO: Paraphrase
\subsection{Augmented Reality}
% "Augmented Reality (AR) has been defined broadly as combining real and computer-generated digital information into the user’s view of the physical and interactive real world in such a way that they appear as one environment, thus providing a bridge between digital information and the physical world" (\cite{Hoellerer2004,Klopfer2007,Vallino1998,Wellner1993} via \cite{Olsson2011b}). 

Application areas of AR can be as broad as ranging from health care \cite{Lui2011} to education \cite{Mannuss2011} to tourism \cite{Mulloni2011}. The topic has gained momentum thanks to the rise in smartphone usage.

% "MAR services were expected to empower people with novel context- sensitive and proactive functionalities, make their activities more efficient, raise awareness of the information related to their surroundings with a very intuitive interface, and provide personally and contextually relevant information that is reliable and up-to-date. Additionally, expectations of offering stimulating and pleasant experiences, such as playfulness, inspiration, liveliness, captivation, and surprise were identified" \cite{Olsson2011b}.

% "Scenarios demonstrating pragmatic relevance were valued more highly than pleasure-oriented scenarios. […] AR was seen to make possible novel interaction possibilities and provide information that was not so easily accessible before.[…] fears of information flood, users’ loss of autonomy, and virtual experiences and information replacing the real were highlighted as negative aspects." \cite{OlssonU}

\subsubsection{History}

\subsubsection{Categories}
AR applications can be categorized in two classes, \textit{AR browsers} and \textit{image recognition-based AR applications} \cite{Olsson2011a}.

\subsubsection{Native apps and web applications}

\subsubsection{WebGL usage}

\subsubsection{Evaluation criteria}
User-oriented issues critical for AR evolution and adoption \cite{Olsson2011b}. "AR research still lacks evaluation methods" \cite{Gandy2010} [12]. Metrics are still very abstract \cite{Olsson2011a}. Obstacles for evaluation the usability of mixed reality systems ("lack of, for example, a common testing platforms and benchmarks")\cite{Bach2004}.
Thus, low user evaluation rates in AR research, though, many evaluate only early tech demos (cf. \cite{Duenster2008}). Some evaluation oriented towards user experience \cite{Bach2004}.

\subsubsection{Impact on Human information behavior }
Human behavior towards information is influenced by new technologies. recommendation agents - reducing the consumer’s information overload and search complexity \cite{Kowatch2010}. acquiring product information in in-store settings has often been linked to consumer decision making and information processing \cite{Karpischek2010,Kowatch2010,Xiao2007}. social aspects of mobile image recognition - attaching digital storytelling to physical products - affective influence \cite{Barthel2010}, e.g. trust, engage consumers to communicate and receive information about products \cite{Karpischek2010}

% TODO Paraphrase
\subsection{UI and UX Research}
"User experience (UX) is regarded as a holistic concept describing the subjective experience resulting from the interaction with a technological product or service. Both instrumental (e.g. utility, usability, and other pragmatic elements) and non- instrumental (e.g. pleasure, appeal, aesthetics, and other hedonic elements) elements are covered in the UX literature \cite{Hassenzahl2006}. A recent ISO standard \cite{FDIS2009} defines UX holistically as “a person's perceptions and responses that result from the use or anticipated use of a product, system or service.” Recently, pleasurable user experience has become the principal goal in the design of novel interactive systems. Hence, the emphasis on UX has moved the design focus from removing usability and functionality problems or other negative factors to offering possibilities for positive and satisfying experiences that exceed the user’s expectations."

Research on human-computer interaction (HCI) has been traditionally been rooted in cognitive psychology, engineering, and computer sciences. Besides these fields, research on emotional factors of design is growing \cite{Norman2002}, which marks a shifts from \textit{usability} to \textit{user experience} analysis.

Academia evaluates user experience oftentimes by looking at the emotional state of the user \cite{Roto2006}, e.g., by letting users fill out pen-and-paper questionnaires during the course of the usage or by capturing user-made video diaries \cite{Csikszentmihalyi1987, Isomursu2004}.




\section{Implementation}
Although data from a gyroscope is accessible with JavaScript and could be therefore used in a WebGL application, we only used data provided by a devices accelerometer, as not all of our test devices had a gyroscope included.

\subsection{iOS WebGLViewer} \label{WebGLViewer}
	As mentioned in section \ref{WebGLoniOS} there is no official WebGL support in a browser on iOS available. Therefore a simple iOS application, called WebGLViewer, was implemented based on the method and instruction published by Nathan de Vries \cite{deVries2011}. WebGLViewer consists of a UIWebView, a reload button and an address bar. The UIWebView was modified to show WebGL content (compare section \ref{WebGLoniOS} and \cite{deVries2011}). 
	
	If the device is shaken while running WebGLViewer the app shows information about the battery consumption. The current battery level, the battery level when the loading of the current page finished as well as the delta of these two values is shown as percentage. In addition the time the website is shown as well as the battery consumption per minute is shown. Unfortunately the battery level is only updated in 5\,\% steps. To get a more accurate result the battery information pop-up will automatically be shown if the battery level changes.
	
	The app WebGLViewer was used to view and evaluate WebGL content on iOS. 

\subsection{Room}

\section{Evaluation}

\subsubsection{Battery and CPU usage}
To measure the battery usage the WebGLViewer app for iOS was used (see section \ref{WebGLViewer}). The device was fully charged. The accordant content was opened in the WebGLViewer app. The device then was unplugged and the content reloaded. The time till the battery level reached 90\,\% was measured by the WebGLViewer. Using Apple Instruments, which is part of Apples Developer Tools, we measured the CPU Activity (Total Activity, Foreground App Activity and Graphics) as well as the relative Energy Usage on a scale from 0 to 20. All tests have been performed on an Apple iPhone 4 with iOS 5.1 installed. We compared the battery usage and CPU activity of our room, the Quake 3 WebGL Demo by Brandon Jones \cite{quakewebgl} and google.com as reference. The results are shown in table \ref{batteryTable}. 

\begin{table*}[tb]
	\begin{centering}
	\begin{tabular}{r|c|c|c|c|c|c}
		\multirow{2}{*}{\textbf{Test Case}} & \multicolumn{3}{c|}{\textbf{Battery and Energy Consumption}}						 & \multicolumn{3}{c}{\textbf{CPU Activity}}					\\
 							&\textbf{Time till 90\,\%} & \textbf{Battery Usage}	& \textbf{Relative Energy Usage} & \textbf{Total} 	& \textbf{Foreground App} 		& \textbf{Graphics}	\\
		\hline
		room.html		   &	36.03 min				& 0.278 $\frac{\%}{min}$	&	17							& 100\,\%			& 64\,\%						& 20\,\%			\\
		Quake 3			   &	34.67 min				& 0.289 $\frac{\%}{min}$	&	16							& 70\,\%			& 60\,\%						& 8\,\%			\\
		google.com		   &	78.93 min				& 0.127	$\frac{\%}{min}$	&	11							& 6\,\%				& 0.2\,\%						& 0.5\,\%		\\
	\end{tabular}
	\caption{Battery consumption and CPU activity of different WebGL applications\label{batteryTable}}
	\end{centering}
\end{table*}

Comparing the two WebGL applications to a normal website like \url{google.com} shows a huge difference in CPU activity. Whereas the overall CPU activity is below 10\,\%, with a value below 1\,\% for Foreground App Activity as well as Graphics Activity, while using \url{google.com}, the overall CPU activity is over 70\,\%, with about 60\,\% for Foreground App Activity and up to 25\,\% for Graphics Activity, while using a WebGL application. 

If we look at the battery consumption there is also a significant difference visible. The battery drains as twice as fast while using a WebGL enabled website compared to \url{google.com}. The difference regarding the Relative Energy Usage is not as big, however we don't know how Apple calculates this value. \url{google.com} still needs less energy compared to the WebGL applications. Also the room.html has a higher value compared to the Quake 3 WebGL Demo, which is consistent with our measurement of the Battery Usage as well as with the Time till the battery level reaches 90\,\%.

\subsection{Frames per second}
% TODO: cite 12 fps human eye
To test the performance of WebGL on different platforms we measured and compared frames per second (FPS) for different applications on different devices. FPS describes the frequency at which images are generated. Below 12 FPS the human eyes is able to recognize the images as different images, for higher FPS a single image can’t be recognized and the images blend together creating motion. So a higher number of FPS creates a more fluid animation and is therefore considered better.

Since the refresh rate of modern flat screens is 60 FPS there is usually no need to calculate more than 60 FPS. % TODO: cite
% The higher the FPS the better, but the number should not exceed 60 (or at least the software %stack should make sure it doesn’t) , since this is the refresh frequency of modern flat screens, % so it would be unnecessary (I had a reference on that, just need to find it).

%TODO: \citel{http://www.webkit.org/blog-files/webgl/SpiritBox.html}.
In table \ref{fpsTable} we compared the frame rate of the implemented room, the Quake 3 WebGL Demo \cite{quakewebgl} and a example of a spinning cube, the “Spirit Box” from \url {webkit.org} on different mobile devices and one powerful desktop computer as reference.

\begin{table*}[tb]
	\begin{centering}
	\begin{tabular}{l|l|l|l|l|l|l}
		\textbf{Device}	& \textbf{Operating System}	& \textbf{Browser}	& \textbf{Launched}	& \textbf{room.html}	& \textbf{Quake 3}	&	\textbf{SpiritBox}	\\
		\hline
		Apple iPad 2				& iOS 5.1			& WebGLViewer	& 2011		& 61 FPS	& 61 FPS	& 60 FPS		\\ 
		HTC EVO 3D					& Android 2.3.4		& Firefox		& 2011		& 8 FPS		& 12 FPS	& N/A $^{1}$	\\ 
		Apple iPhone 4				& iOS 5.1			& WebGLViewer	& 2010		& 40 FPS	& 29 FPS	& 43 FPS		\\
		Apple iPod Touch (4th Gen.)	& iOS 5.1			& WebGLViewer	& 2010		& 21 FPS	& 25 FPS	& 43 FPS		\\ 
%		HTC Desire					& Android 2.2.2		& Opera v12		& 2010		& 6 FPS 	&			&				\\ % room black == not working ???
		HTC Desire					& Android 2.2.2		& Firefox		& 2010		& 2 FPS 	& N/A		& 8 FPS			\\ % Mozilla == Firefox ??
		Apple iPhone 3GS			& iOS 5.1			& WebGLViewer	& 2009		&	36 FPS		&	27 FPS	&	60 FPS			\\
		\hline
		Reference Laptop $^{2}$			& 	Mac OS X 10.7.3			& 	Google Chrome	&	2010	&	34 FPS	&	36 FPS	&		50 FPS	\\
		Reference Computer $^{3}$		& 	Mac OS X 10.7.3			& 	Google Chrome	&	2006	&	58 FPS	&	59 FPS	&		85 FPS	\\
	\end{tabular}
	\caption{Frames per Second (FPS) of different WebGL applications on different devices\label{fpsTable}
	\\$^{1}$ \textit{The FPS value is constantly alternating between values in the range from 10 FPS up to over 200 FPS, making it impossible to determine a realistic value.}
	\\$^{2}$ \textit{Reference Laptop: MacBook 2010, Mac OS X 10.7.3, 2.26 GHz Intel Core 2 Duo, 4 GB 1067 MHz DDR3 RAM}}
	\\$^{3}$ \textit{Reference Computer: MacBook 2006, Mac OS X 10.7.3, 2x 2.0 GHz Dual-Core Intel Xeon, 6 GB 667 MHz DDR2 RAM}}
	\end{centering}
\end{table*}

\begin{table*}[tb]
	\begin{centering}
	\begin{tabular}{l|l|l|l|l|l|l|l}
		\textbf{Lesson}								& \textbf{Reference$^{1}$}& \textbf{Desire}		& \textbf{EVO 3D}		& \textbf{iPhone 4}		& \textbf{iPhone 3GS}	& \textbf{iPad 2}	& \textbf{iPod Touch}\\
		\hline
		3 - Simple animations						& 59 FPS		& 13 FPS		& 16 FPS		& 57 FPS		& 57 FPS		& 58 FPS	& 40 FPS				\\
		4 - 3D animations							& 58 FPS		& 12 FPS		& 19 FPS		& 57 FPS		& 57 FPS		& 58 FPS	& 57 FPS				\\
		5 - Textures								& 59 FPS		& 13 FPS		& 16 FPS		& 40 FPS		& 57 FPS		& 58 FPS	& 40 FPS				\\
		6 - Texture filters and keyboard			& 59 FPS		& 12 FPS		& 18 FPS		& 40 FPS		& 57 FPS		& 58 FPS	& 40 FPS				\\
		7 - Basic Lighting							& 59 FPS		& 12 FPS		& 17 FPS		& 57 FPS		& 57 FPS		& 58 FPS	& 57 FPS				\\
		8 - Transparency and blending				& 59 FPS		& 11 FPS		& 16 FPS		& 58 FPS		& 57 FPS		& 58 FPS	& 58 FPS				\\
		9 - Particles								& 59 FPS		&  8 FPS		& 17 FPS		& 39 FPS		& 31 FPS		& 60 FPS	& 40 FPS				\\
		10 - Loading a map							& 58 FPS		& 11 FPS		& 15 FPS		& 57 FPS		& 57 FPS		& 58 FPS	& 40 FPS				\\
		11 - Sphere and rotation					& 59 FPS		& 12 FPS		& 15 FPS		& 40 FPS		& 57 FPS		& 58 FPS	& 57 FPS				\\
		12 - Point lighting							& 58 FPS		& 11 FPS		& 15 FPS		& 40 FPS		& 57 FPS		& 58 FPS	& 40 FPS				\\
		13 - Per-fragment lighting					& 59 FPS		& 10 FPS		& 17 FPS		& 36 FPS		& 40 FPS		& 58 FPS	& 39 FPS				\\
		14 - Specular highlights and JSON model		& 58 FPS		&  9 FPS		& 15 FPS		& 35 FPS		& 39 FPS		& 58 FPS	& 36 FPS				\\
		15 - Specular maps							& 58 FPS		&  8 FPS		& 16 FPS		& 22 FPS $^{2}$	& 24 FPS $^{2}$	& 58 FPS $^{2}$	& 22 FPS $^{2}$ 		\\
		16 - Render to texture						& 58 FPS		&  6 FPS		& 16 FPS		& 22 FPS		& 22 FPS		& 58 FPS	& 22 FPS				\\
	\end{tabular}
	\caption{Frames per Second (FPS) using the lessons from the learning WebGL tutorial on different devices\label{lessonsTable}
	\\\textit{$^{1}$ Reference Computer: MacBook 2010, Mac OS X 10.7.3, 2.26 GHz Intel Core 2 Duo, 4 GB 1067 MHz DDR3 RAM}
	\\\textit{$^{2}$ Rendering results aren't looking as expected}
	}
	\end{centering}
\end{table*}

\section{Conclusion}
The conclusion goes here.





\appendices
\section*{Installing WebGLViewer}
To install the WebGLViewer using the provided source code on an iOS device a membership in Apple's iOS Developer Program is needed. However compiling and running the WebGLViewer in the iOS Simulator is possible using XCode. As alternative, a registered developer can provide you a binary, which is build for your specific device. We used \url{testflight.com} to distribute such binaries. To request a binary you need to create an account on \url{http://bit.ly/zWoZQJ} and register your device as soon as you are accepted.

\section*{Appendix B}
Appendix two text goes here.

\begin{lstlisting}[label=code:hello_world, caption={Hello World Code Snippet}]
System.out.println("Hello World!");
\end{lstlisting}

\ifCLASSOPTIONcaptionsoff
  \newpage
\fi



\bibliographystyle{IEEEtran}
\bibliography{bib}

\end{document}


